% docs/slides.tex
\documentclass[aspectratio=169]{beamer}
\usepackage{xeCJK}
\setCJKmainfont{Source Han Sans TW} % 保持與報告一致的字體
\usepackage{booktabs}
\usepackage{graphicx}

% 設定主題 (推薦用簡潔的 metropolis 或 default)
\usetheme{Madrid}
\usecolortheme{beaver}

\title{經典風險因子模型之比較與解釋力分析}
\subtitle{金融科技導論期末報告 (M-3)}
\author{鍾詠傑 (41173058H)}
\institute{國立臺灣師範大學}
\date{\today}

\begin{document}

% 封面
\frame{\titlepage}

% 大綱
\begin{frame}{報告大綱}
    \tableofcontents
\end{frame}

% 1. 研究動機
\section{研究動機與選股}
\begin{frame}{研究動機與選股策略}
    \begin{itemize}
        \item \textbf{目標}:比較 CAPM、FF3、FF5 在消費零售業的解釋力。
        \item \textbf{選股 (10檔)}:
        \begin{itemize}
            \item \textbf{必需消費 (4檔)}:Walmart, Costco, Coca-Cola, Pepsi
            \item \textbf{非必需消費 (5檔)}:Nike, Starbucks, McDonald's, Home Depot, Target
            \item \textbf{科技零售 (1檔)}:Amazon (作為成長股對照組)
        \end{itemize}
        \item \textbf{研究問題}:每個新增因子解決了什麼問題?
    \end{itemize}
\end{frame}

% 2. 累積報酬圖表 (直接引用產出的圖片)
\section{績效表現}
\begin{frame}{投資組合累積報酬率 (2019-2023)}
    \begin{figure}
        \centering
        \includegraphics[width=0.85\textwidth]{../output/images/cumulative_returns.png}
        \caption{Amazon 展現高成長與高波動,傳統零售股相對穩健。}
    \end{figure}
\end{frame}

% 3. 模型比較 (直接引用產出的表格)
\section{模型解釋力分析}
\begin{frame}{三種模型的演進}
    \begin{block}{CAPM → FF3 → FF5}
    \begin{itemize}
        \item \textbf{CAPM}:只考慮市場風險 ($\beta$)
        \item \textbf{FF3}:新增規模 (SMB) 與價值 (HML) 因子
        \item \textbf{FF5}:再加入獲利能力 (RMW) 與投資風格 (CMA)
    \end{itemize}
    \end{block}
    \vspace{0.5cm}
    \textbf{核心問題}:為何需要這些新因子?
    \begin{itemize}
        \item CAPM 無法解釋小型股、價值股、高品質股的超額報酬
        \item FF5 成功捕捉「品質溢價」與「投資風格差異」
    \end{itemize}
\end{frame}

\begin{frame}{模型解釋力比較 ($R^2$)}
    \begin{table}
        \centering
        \tiny % 縮小字體以放入投影片
        \input{../output/tables/r2_comparison.tex}
        \caption{FF5 模型顯著提升了 Amazon 與 Costco 的解釋力。}
    \end{table}
    
    \textbf{關鍵發現}:
    \begin{itemize}
        \item Amazon $R^2$: 0.43 (CAPM) → 0.60 (FF5) \textcolor{red}{(+40\%)}
        \item 平均提升:CAPM 到 FF5 約 15-20\% 解釋力
    \end{itemize}
\end{frame}

% 4. 因子分析
\section{因子顯著性}
\begin{frame}{每個因子解決了什麼問題?}
    \begin{itemize}
        \item \textbf{SMB \& HML (FF3新增)}:
        \begin{itemize}
            \item 解決「規模效應」與「價值溢價」問題
            \item 例:KO, MCD 的超額報酬不再是異常現象
        \end{itemize}
        \vspace{0.3cm}
        \item \textbf{RMW (FF5新增)}:
        \begin{itemize}
            \item 解決「高品質公司的持續超額報酬」問題
            \item 例:Costco (0.45)、Home Depot (0.58) 的品質溢價
        \end{itemize}
        \vspace{0.3cm}
        \item \textbf{CMA (FF5新增)}:
        \begin{itemize}
            \item 解決「成長股 vs. 成熟股的風險差異」問題
            \item 例:Amazon (-1.01) 激進 vs. Pepsi (0.69) 保守
        \end{itemize}
    \end{itemize}
\end{frame}

\begin{frame}{Fama-French 五因子係數分析}
    \begin{columns}
        \column{0.5\textwidth}
        \textbf{關鍵發現:}
        \begin{itemize}
            \item \textbf{SMB}:全為負值,符合大型股特徵。
            \item \textbf{RMW}:Costco (0.45)、HD (0.58) 顯著為正 → 高品質護城河。
            \item \textbf{CMA}:Amazon (-1.01) vs. Pepsi (0.69) → 激進擴張 vs. 保守穩健。
        \end{itemize}

        \column{0.5\textwidth}
        \begin{figure}
            \includegraphics[width=\textwidth]{../output/images/factor_corr.png}
            \caption{因子相關性熱圖:多重共線性低}
        \end{figure}
    \end{columns}
\end{frame}

% 5. 結論
\section{結論}
\begin{frame}{結論}
    \begin{block}{總結}
        \begin{enumerate}
            \item \textbf{模型優勢}:FF5 能有效解釋 CAPM 無法捕捉的「品質 (Quality)」與「成長 (Growth)」溢酬。
            \item \textbf{產業洞察}:零售業巨頭普遍具有高 RMW 特徵。
            \item \textbf{投資應用}:建議在升息環境下配置高 CMA (保守投資) 與高 RMW (高獲利) 的防禦型股票。
        \end{enumerate}
    \end{block}
\end{frame}

\begin{frame}
    \centering \Huge 謝謝聆聽 \\ \large Q \& A
\end{frame}

\end{document}
