\documentclass[12pt, a4paper]{article}
\usepackage[utf8]{inputenc}
% 若需要中文字體 (建議使用 XeLaTeX 編譯)
\usepackage{xeCJK}
\setCJKmainfont{Source Han Sans TW} % 這裡請換成你系統有的字體,如 "Microsoft JhengHei" 或 "Noto Sans TC"

\usepackage{geometry}
\usepackage{graphicx}
\usepackage{booktabs} % 用於漂亮的表格線條
\usepackage{amsmath}
\usepackage{float}
\usepackage{hyperref}

\geometry{left=2.5cm, right=2.5cm, top=2.5cm, bottom=2.5cm}

\title{期末報告:M-3 消費與零售通路組合之資產定價分析}
\author{你的姓名 \\ 學號:XXXXXXXX}
\date{\today}

\begin{document}

\maketitle

\section{選股與特徵整理}
本研究選擇了 10 檔代表性的消費與零售股票,包含必需性消費(如 Walmart, Costco)與非必需性消費(如 Nike, Starbucks),並加入 Amazon 作為科技零售的對照組。

\begin{figure}[H]
    \centering
    \includegraphics[width=0.9\textwidth]{../output/images/cumulative_returns.png}
    \caption{投資組合累積報酬率 (2020-2023)}
    \label{fig:returns}
\end{figure}

\section{模型架構}
本研究比較三種經典資產定價模型。以 Fama-French 五因子模型為例,其回歸方程式如下:
\begin{equation}
R_{it} - R_{ft} = \alpha_i + \beta_i(R_{Mt} - R_{ft}) + s_i SMB_t + h_i HML_t + r_i RMW_t + c_i CMA_t + \epsilon_{it}
\end{equation}

\section{實證結果分析}

\subsection{模型解釋力比較}
透過比較調整後 $R^2$,我們可以觀察新增因子對風險解釋力的提升。表 \ref{tab:r2_compare} 顯示了三個模型的比較結果。
從數據中可以發現,加入 FF3 與 FF5 因子後,模型的解釋力普遍提升。
特別是 \textbf{Amazon (AMZN)},其 $R^2$ 從 CAPM 的 0.43 提升至 FF5 的 0.60,顯示單純的市場因子無法解釋其高成長與投資屬性。
此外,\textbf{Pepsi (PEP)} 與 \textbf{Walmart (WMT)} 在加入因子後解釋力也有顯著增加,證實了這些股票具有明顯的風格特徵。

% 這裡是自動引入 Python 生成的表格
\input{../output/tables/r2_comparison.tex}

\subsection{因子顯著性分析}
Fama-French 五因子的具體係數估計如表 \ref{tab:ff5_params} 所示。
\begin{itemize}
    \item \textbf{規模因子 (SMB)}: 由於選股多為大型股,預期 $s_i$ 多為負值。
    \item \textbf{獲利能力因子 (RMW)}: 幾乎所有選股的 RMW 係數皆為正值,其中 \textbf{Home Depot (HD, 0.58)} 與 \textbf{Costco (COST, 0.45)} 係數最高,反映其強大的護城河與高 ROE 特徵,屬於典型的「品質股 (Quality Stocks)」。
    \item \textbf{投資風格因子 (CMA)}: \textbf{Amazon (AMZN)} 的 CMA 係數為顯著負值 (-1.01),反映其積極擴張與高資本支出的特性(Aggressive Investment)。相對地,\textbf{Pepsi (PEP, 0.69)} 與 \textbf{Walmart (WMT, 0.58)} 則呈現保守投資特徵,符合其防禦型股票的定位。
\end{itemize}

% 這裡引入第二張表格
\input{../output/tables/ff5_factors.tex}

\section{結論}
實證結果顯示,FF5 模型在消費零售類股的解釋力優於 CAPM 與 FF3。特別是加入 RMW 因子後,有效解釋了高品質零售股(Quality Stocks)的超額報酬來源。

\end{document}
