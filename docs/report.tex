\documentclass[12pt, a4paper]{article}
\usepackage[utf8]{inputenc}
% 若需要中文字體 (建議使用 XeLaTeX 編譯)
\usepackage{xeCJK}
\setCJKmainfont{Source Han Sans TW} % 這裡請換成你系統有的字體,如 "Microsoft JhengHei" 或 "Noto Sans TC"

\usepackage{geometry}
\usepackage{graphicx}
\usepackage{booktabs} % 用於漂亮的表格線條
\usepackage{amsmath}
\usepackage{float}
\usepackage{hyperref}
\usepackage{caption} % 讓標題更好看

\geometry{left=2.5cm, right=2.5cm, top=2.5cm, bottom=2.5cm}

\title{金融科技導論期末報告:經典風險因子模型之比較與解釋力分析}
\author{鍾詠傑 \\ 學號:41173058H}
\date{\today}

\begin{document}

\maketitle

\section{選股與特徵整理}
本研究選擇了 10 檔代表性的消費與零售股票,涵蓋必需性消費、非必需性消費以及科技零售。表 \ref{tab:stock_list} 整理了投資組合的基本特徵。

% 股票清單與特徵表格
\begin{table}[H]
    \centering
    \caption{投資組合選股與基本特徵}
    \label{tab:stock_list}
    \begin{tabular}{llll}
        \toprule
        \textbf{代號} & \textbf{公司名稱} & \textbf{產業分類} & \textbf{預期屬性} \\
        \midrule
        WMT & Walmart & 必需消費 (超市) & 價值、防禦型 \\
        COST & Costco & 必需消費 (量販) & 高周轉、高品質 \\
        KO & Coca-Cola & 必需消費 (飲料) & 高股息、低波動 \\
        PEP & PepsiCo & 必需消費 (飲料) & 穩健成長 \\
        MCD & McDonald's & 非必需 (餐飲) & 全球化連鎖 \\
        SBUX & Starbucks & 非必需 (餐飲) & 成長型餐飲 \\
        NKE & Nike & 非必需 (服飾) & 品牌高溢價 \\
        HD & Home Depot & 非必需 (家居) & 房市週期連動 \\
        TGT & Target & 非必需 (百貨) & 價值型 \\
        AMZN & Amazon & 非必需 (電商) & 高成長、科技屬性 \\
        \bottomrule
    \end{tabular}
\end{table}

圖 \ref{fig:returns} 展示了投資組合在 2019 至 2023 年間的累積報酬率表現。可以看出 Amazon (AMZN) 波動劇烈,而傳統必需消費股(如 KO, PEP)則相對穩健。

\begin{figure}[H]
    \centering
    \includegraphics[width=0.9\textwidth]{../output/images/cumulative_returns.png}
    \caption{投資組合累積報酬率 (2019-2023)}
    \label{fig:returns}
\end{figure}

\section{模型架構}
本研究由簡入繁,比較三種經典資產定價模型:

\subsection{CAPM (資本資產定價模型)}
僅考慮市場風險因子:
\begin{equation}
R_{it} - R_{ft} = \alpha_i + \beta_i(R_{Mt} - R_{ft}) + \epsilon_{it}
\end{equation}
其中 $R_{it}$ 為資產 $i$ 在時間 $t$ 的報酬率,$R_{ft}$ 為無風險利率,$R_{Mt}$ 為市場報酬率,$\beta_i$ 為市場風險係數。

\subsection{Fama-French 三因子模型 (FF3)}
加入規模 (SMB) 與價值 (HML) 因子,解釋小型股與價值股效應:
\begin{equation}
R_{it} - R_{ft} = \alpha_i + \beta_i(R_{Mt} - R_{ft}) + s_i SMB_t + h_i HML_t + \epsilon_{it}
\end{equation}
其中 $SMB_t$ (Small Minus Big) 為規模因子,$HML_t$ (High Minus Low) 為價值因子(帳面市值比)。

\subsection{Fama-French 五因子模型 (FF5)}
進一步加入獲利能力 (RMW) 與投資風格 (CMA) 因子,以解決 FF3 無法解釋高品質與高成長股回報的問題:
\begin{equation}
R_{it} - R_{ft} = \alpha_i + \beta_i(R_{Mt} - R_{ft}) + s_i SMB_t + h_i HML_t + r_i RMW_t + c_i CMA_t + \epsilon_{it}
\end{equation}
其中 $RMW_t$ (Robust Minus Weak) 為獲利能力因子,$CMA_t$ (Conservative Minus Aggressive) 為投資風格因子。

\section{實證結果分析}

\subsection{模型解釋力比較}
表 \ref{tab:r2_compare} 顯示了三個模型的調整後 $R^2$ 比較。
數據顯示,從 CAPM 到 FF5,模型的解釋力普遍提升。
\begin{itemize}
    \item \textbf{科技成長股的改善}:\textbf{Amazon (AMZN)} 的 $R^2$ 從 CAPM 的 0.43 顯著提升至 FF5 的 0.60。這暗示了單純的市場因子無法解釋其高成長與激進投資的風險屬性。
    \item \textbf{防禦型類股的改善}:\textbf{Pepsi (PEP)} 與 \textbf{Walmart (WMT)} 在加入因子後解釋力也有增加,證實了這些股票的收益部分來自於其「價值」與「保守投資」特徵。
    \item \textbf{高品質零售的改善}:\textbf{Costco (COST)} 與 \textbf{Coca-Cola (KO)} 的解釋力也有顯著提升,FF5 模型成功捕捉了其高獲利能力的溢價。
\end{itemize}

\input{../output/tables/r2_comparison.tex}

\subsection{因子顯著性與經濟意義}
圖 \ref{fig:corr} 為五因子的相關性熱圖,顯示因子間相關性低,模型多重共線性問題不大。這驗證了 Fama-French 五因子模型的獨立性設計。

\begin{figure}[H]
    \centering
    \includegraphics[width=0.7\textwidth]{../output/images/factor_corr.png}
    \caption{Fama-French 五因子相關性分析}
    \label{fig:corr}
\end{figure}

表 \ref{tab:ff5_params} 展示了 FF5 模型的具體係數估計,我們觀察到顯著的產業特徵:
\begin{itemize}
    \item \textbf{規模因子 (SMB)}: 由於本研究選股皆為大型權值股,預期 $s_i$ 多為負值或接近零,實證結果符合預期。例如 PepsiCo (PEP) 的 SMB 係數為 -0.35,反映其大型股特性。
    \item \textbf{價值因子 (HML)}: \textbf{Costco (COST, -0.44)} 與 \textbf{Amazon (AMZN, -0.33)} 的 HML 係數顯著為負,證實了其成長股屬性;而 \textbf{McDonald's (MCD, 0.09)} 與 \textbf{Coca-Cola (KO, 0.12)} 則展現價值股特徵。
    \item \textbf{獲利能力因子 (RMW)}: \textbf{Home Depot (HD, 0.58)} 與 \textbf{Costco (COST, 0.45)} 的係數顯著為正。這解釋了為什麼這些高估值(高 P/E)的股票仍能提供高回報——市場在獎勵其強大的護城河與高 ROE(品質溢價)。這是 CAPM 與 FF3 無法捕捉的關鍵因素。
    \item \textbf{投資風格因子 (CMA)}: 這是區分 Amazon 與傳統零售的關鍵。\textbf{Amazon (AMZN)} 的 CMA 係數為顯著負值 (-1.01),反映其積極擴張(Aggressive Investment)的商業策略;而 \textbf{Pepsi (PEP, 0.69)} 與 \textbf{Walmart (WMT, 0.58)} 則呈現保守投資特徵 (Conservative),符合其成熟企業的定位。
\end{itemize}

\input{../output/tables/ff5_factors.tex}

\subsection{每個因子解決了什麼問題?}
\begin{itemize}
    \item \textbf{SMB 與 HML (FF3 新增)}:解決了 CAPM 無法解釋的「規模效應」與「價值溢價」問題。例如,傳統價值股(如 KO, MCD)的超額報酬在 CAPM 下會被視為異常,但 FF3 透過 HML 因子合理化了這種現象。
    \item \textbf{RMW (FF5 新增)}:解決了「高品質公司為何能持續獲得超額報酬」的問題。在消費零售業,高 ROE 與強大品牌力(如 Costco, Home Depot)的公司,其股價表現優於同業,但 FF3 無法解釋這種「品質溢價」。
    \item \textbf{CMA (FF5 新增)}:解決了「為何激進投資的公司(如 Amazon)與保守投資的公司(如 Pepsi)有不同的風險報酬結構」的問題。這對於區分成長股與成熟股至關重要。
\end{itemize}

\section{結論}
本研究比較了 CAPM、FF3 與 FF5 在消費零售類股的解釋力。結論如下:
\begin{enumerate}
    \item \textbf{模型效力}:FF5 模型顯著優於 CAPM,特別是對於風格鮮明的股票(如高成長的 Amazon 或高品質的 Costco)。平均而言,FF5 的調整後 $R^2$ 比 CAPM 高出 15-20\%。
    \item \textbf{因子洞察}:實證發現,零售業巨頭普遍具有顯著的 \textbf{RMW (獲利能力)} 曝險。這意味著在分析零售股時,除了市場風險外,應高度重視公司的營運效率與品質因子。
    \item \textbf{CAPM 的不足}:單一市場因子難以捕捉「激進投資」與「價值防禦」的差異,導致對 Amazon 或 Coca-Cola 的定價誤差較大。具體而言,CAPM 對 Amazon 的 $R^2$ 僅有 0.43,無法解釋其 57\% 的報酬變異。
    \item \textbf{投資應用}:基於模型分析,若要建構防禦型組合,應選擇 HML 和 RMW 高的股票(如 KO, PEP);若追求成長,則應關注 CMA 為負且 RMW 為正的股票(高品質成長股)。
\end{enumerate}

\end{document}
